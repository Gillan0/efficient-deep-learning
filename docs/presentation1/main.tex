\documentclass{beamer}
\usepackage{graphicx} % Required for inserting images
\usepackage{csvsimple}
\usepackage{booktabs}
\usepackage{caption}
\usepackage{subcaption}
\usepackage{amsmath}
\usetheme{Berlin}

\usepackage{verbatim}


\setbeamertemplate{caption}[numbered]
\setbeamertemplate{blocks}[rounded][shadow=true]
\setbeamertemplate{footline}{%
  \leavevmode%
  \hbox{%
  \begin{beamercolorbox}[wd=.9\paperwidth,ht=2.5ex,dp=1.125ex,leftskip=.3cm,rightskip=.3cm]{author in head/foot}%
    \usebeamerfont{author in head/foot}%
    \insertshorttitle\hspace*{2em}%
    \insertframenumber\,/\,\inserttotalframenumber\hfill%
  \end{beamercolorbox}}%
  \vskip0pt%
}


\title{Transformers in Computer Vision and Hyperparameter exploration on ResNet
}
\subtitle{Efficient Deep Learning}
\author{GILLARD Antonino}
\date{2026/02/17}

\begin{document}

\maketitle

\section{Paper presentation}
\begin{frame}{Paper Introduction}
  

\centering
\vspace{0.3cm}

{\Large \bfseries
An Image is Worth 16×16 Words : 

Transformers for Image Recognition at Scale\\
}

\vspace{0.2cm}

{\normalsize
Dosovitskiy et al., 2021
}

\vspace{0.3cm}

\begin{block}{Context}
  \begin{itemize}
    \item Image classification is mostly done by CNN
    \item Transformers : NLP tasks
    \item Attempts at using self attention mechanisms in image classification do not scale
  \end{itemize}
\end{block}

\textbf{ResNet like networks are still state of the art for image classification}


\end{frame}

\begin{frame}{Main Idea : Treat Images like texts in Transformers}
\begin{block}{Transformers in NLP}
  \begin{itemize}
    \item Texts are split into \textbf{tokens} : 2-4 character long pieces
    \item Tokens are passed through an embedding (incl. positions)
    \item Pass them through transformers and then output
  \end{itemize}
\end{block}

\begin{block}{Vision Transformer (ViT)}
  \begin{itemize}
    \item Split images in 16x16 patches 
    \item Patches are flattened then passed through an embedding (incl. 2D position in the image)
    \item Pass them through transformers and then classification
  \end{itemize}
\end{block}

\end{frame}

\begin{frame}{ViT representation}
  \begin{figure}
    \centering
    \hfill
    \includegraphics[width=0.95\textwidth]{ressources/vit-overview.png}
    \caption{ViT Model Overview}
  \end{figure} 
\end{frame}

\begin{frame}{Performance Benchmarks}
\begin{figure}
  \centering
  \small
  % First graph
  \begin{subfigure}[b]{0.75\textwidth}
      \centering
      \includegraphics[width=\textwidth]{ressources/results-1.png} % replace with your file
  \end{subfigure}
  \hfill
  % Second graph
  \begin{subfigure}[b]{0.75\textwidth}
      \centering
      \includegraphics[width=\textwidth]{ressources/results-2.png} % replace with your file
  \end{subfigure}
  \caption{Model evalutations}
\end{figure}
\end{frame}

\begin{frame}{Paper Conclusion}
  \begin{block}{Model performance}
    \begin{itemize}
      \item Reaches state of the art performance when trained on large datasets
      \item CNN models such as BiT (ResNet) still outperform when trained on smaller datasets
    \end{itemize}
    Good potential for image transformers but they are still inefficient. 
  \end{block}

  \begin{block}{There is still some work to be done ...}
    The paper only focuses on one task : \textbf{Image Classification}.
    Although results are promising in that field, work still has to be done on other tasks (e.g. Detection)
  \end{block}
  
\end{frame}

\section{Hyperparameter exploration}

\begin{frame}{Hyperparameter exploration} 

\begin{columns}[T] 
    \begin{column}{0.45\textwidth}
        \begin{block}{ResNet18}
            \begin{itemize}
              \item 18-layer CNN 
              \item Residual connections
              \item Image classification
            \end{itemize}
        \end{block}
    \end{column}
    
    \begin{column}{0.45\textwidth}
        \centering
        \includegraphics[height=2cm]{ressources/ResBlock.png} 
        \captionof{figure}{ResNet architecture}
    \end{column}
\end{columns}


Focus of Hyperparameter exploration strategy :
\textbf{Learning Rate}

\begin{table}[h]
  \centering
  \begin{tabular}{|c|c|c|c|c|}
  \hline
  \textbf{Learning Rate} & 0.1 & 0.01 & 0.001 \\ \hline
  \textbf{Testing Loss} & 0.67 & 0.54 & 0.43 \\ \hline
  \end{tabular}
  \caption{First runs by modifying LR. ResNet18 model with SGD optimizer (weight decay : $5\times 10^{-4}$) and CosineAnnealing of period 20}
\end{table}
\end{frame}

\begin{frame}{Optimizer and Scheduler Configurations}

\begin{table}[h]
  \centering
  \begin{tabular}{|c|c|c|c|}
  \hline
  \textbf{Weight Decay} & \textbf{LR} & \textbf{Loss Function} & \textbf{Epoch} \\ \hline
  $5\times 10^{-6}$ & $10^{-3}$ & Cross Entropy & $\approx 25$ \\ \hline
  \end{tabular}
  \caption{Common parameters across all trained models}
\end{table}


\begin{table}[h]
  \centering
  \begin{tabular}{|c|c|c|c|c|}
  \hline
  \textbf{Optimizer} & \textbf{LR} & \textbf{Momentum} & \textbf{Scheduler} & \textbf{Patience} \\ \hline
  SGD & $10^{-3}$ & 0.9 & CosineAnnealingLR & N/A \\\hline
  SGD & $10^{-3}$ & 0.9 & ReduceLROnPlateau & 5 \\ \hline
  Adam & $10^{-3}$ & N/A & ReduceLROnPlateau & 5 \\ \hline
  \end{tabular}
  \caption{Differences between trained models}
  \label{tab:model_diff}
\end{table}

\end{frame}

\begin{frame}{Training Loss Evolution}
    \centering
    \begin{figure}
        % Adam
        \begin{subfigure}[b]{0.3\textwidth}
            \centering
            \includegraphics[width=\textwidth]{ressources/adam.png}
            \caption{Adam}
        \end{subfigure}
        \hfill
        % SGD Cosine
        \begin{subfigure}[b]{0.3\textwidth}
            \centering
            \includegraphics[width=\textwidth]{ressources/cosine.png}
            \caption{Cosine (SGD)}
        \end{subfigure}
        \hfill
        % SGD Plateau
        \begin{subfigure}[b]{0.3\textwidth}
            \centering
            \includegraphics[width=\textwidth]{ressources/sgd-plateau.png}
            \caption{ReduceLR (SGD)}
        \end{subfigure}
        \vspace{0.3cm}
        \caption{Training loss per epoch for different optimizers and schedulers.}
    \end{figure}

    \begin{itemize}
      \item Start to see some overfitting
      \item However, test accuracy went up during the plateau (+3\%)
    \end{itemize}
\end{frame}


\begin{frame}{Model Accuracy}
  \begin{figure}
    \hfill
    \centering
    \includegraphics[width=\textwidth]{ressources/accuracy_vs_params.png}
    \caption{Accuracy (\%) of each model per number of parameters}
  \end{figure}
\end{frame}

\end{document}